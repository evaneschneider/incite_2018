\documentclass[11pt,letterpaper,english]{article}
\usepackage[T1]{fontenc} % Standard package for selecting font encodingsamely titles, headers and contents
\usepackage{txfonts} % makes spacing between characters space correctly
\usepackage{xcolor} % Driver-independent color extensions for LaTeX and pdfLaTeX.
%\usepackage[pagestyles,raggedright]{titlesec} % related with sections—n
%\usepackage{blindtext} % To create text
\usepackage{fancyhdr} % header footer placement

\usepackage[top=1in, bottom=1in, left=1in, right=1in] {geometry} % Margins
\usepackage{graphicx}  % Essential for adding images to you document.

\usepackage{sectsty}
\sectionfont{\large}
\subsectionfont{\normalsize}
\subsubsectionfont{\normalsize \it}

\usepackage{caption}
\captionsetup{labelsep=period}


\pagestyle{fancy} % allows you to use the header and footer commands

\raggedright
\begin{document}

\setlength{\parindent}{0in} % Amount of indentation at the first line of a paragraph.

\pagestyle{fancy} \lhead{Revealing the Physics of Galactic Winds with Petascale GPU Simulations} \rhead{Brant Robertson} \renewcommand{%
\headrulewidth}{0.0pt}

\begin{center}
\bf \large {PROJECT STATUS SUMMARY} \\
{\bf \small {\em (Must not exceed 1 page.)}}
\end{center}


%\begin{flushleft}
%\bigskip

\textbf{Title (\emph{80 characters max; strictly enforced})}: Revealing the Physics of Galactic Winds with Petascale GPU Simulations \smallskip

\textbf{PI and Co-PI(s)}: Brant Robertson, Evan Schneider \smallskip

\textbf{Applying Institution/Organization}: University of California, Santa Cruz; Princeton University \smallskip

\textbf{Number of Processor Hours Requested}: 58,000,000 \smallskip

\textbf{Amount of Storage Requested}: 160 TB \smallskip

\textbf{Status Summary ({\emph{May use the remainder of page}}):} \\

%Briefly summarize the goals of the project. It is unnecessary to repeat the executive summary from the original proposal. The project status summary should include an overview of the achievements to date. Industry organizations should also summarize the economic or strategic business impact of the accomplishments to date.\\
\vspace{.15in}
The INCITE project "Revealing the Physics of Galactic Winds with Petascale GPU Simulations" has three primary research objectives: (a) to simulate galactic outflows with numerical models that allow for supersonic wind velocities; (b) to quantify the importance of radiative cooling for the multiphase structure of observed galactic outflows; and (c) to determine the mass and energy coupling of ISM gas to supernova-driven outflows. At the close of Semester 1, we have successfully achieved the first of these objectives, and are well on the way to achieving the second. The third and final objective is the focus of our Year 2 renewal proposal.
~\\~\\
\textit{Semester 1 Achievements}. To successfully accomplish Research Objective A, our project has progressed through several anticipated Research Milestones. Tasks completed in Semester 1 include:  adding the capability to model static gravitational potentials to our GPU-based hydrodynamics simulation code \textit{Cholla}; developing rotationally-supported models for isothermal gas disks that match the observed properties of both Milky-Way-like systems, as well as the starburst galaxy M82; developing a model for a pressure-supported halo of gas that is stable on gigayear timescales; and developing a supernova-feedback scheme that drives global outflows with radial profiles that match analytic predictions. All of these preparatory efforts were completed in accordance with our project timeline, and set the stage for the first of our production simulations run at the beginning of Semester 2.
~\\~\\
\textit{Semester 2 Achievements}. At the start of Semester 2, we ran the first of our production simulations, a 17-billion-cell simulation of a disk galaxy with a galactic wind. The resolution of the simulation is <5 pc everywhere in the simulation volume, making it by far the largest simulation of an isolated disk galaxy ever performed (to our knowledge). The properties of the generated galactic wind match those predicted for a non-radiatively-cooling outflow \textit{to high precision}, making it an excellent control for the three radiatively-cooling production simulations to be carried out in the remainder of Semester 2. This simulation also fulfills Research Objective A, as it demonstrates that with the correct simulation geometry, supernova feedback does generate supersonic outflows. Early results from the radiatively-cooling simulations indicate that gas in outflows with appropriate parameters can cool rapidly, so we anticipate definitively answering Research Objective B by the close of the second semester.
~\\~\\
\textit{Project Plans for Year 2}. Year 2 will be devoted to the third of our Research Objectives. In order to quantify the mass and energy coupling of ISM gas, we will increase the physical realism of our current supernova outflow generation scheme. By testing two feedback prescriptions in two production simulations, we will determine how the exact feedback model used can influence the mass-loading and energy content of the outflow - parameters that are set manually in our Year 1 simulations. A simulation maintaining $\sim$2 pc-resolution over 10kpc scales will be the culmination of the project, allowing us to test feedback with winds generated by discrete supernova events in the ISM and tying together the scientific achievements accomplished over all four semesters of the project.

%\end{flushleft}

\end{document}
