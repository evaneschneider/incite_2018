\documentclass[11pt,letterpaper,english]{article}
\usepackage[T1]{fontenc} % Standard package for selecting font encodings
\usepackage{txfonts} % makes spacing between characters space correctly
\usepackage{xcolor} % Driver-independent color extensions for LaTeX and pdfLaTeX.
%\usepackage{blindtext} % To create text
%\usepackage{mdwlist} % mdwlist for compact enumeration/list items 
%\usepackage[pagestyles]{titlesec} % related with sections—namely titles, headers and contents
\usepackage{fancyhdr} % header footer placement

\usepackage[top=1in, bottom=1in, left=1in, right=1in] {geometry} % Margins
\usepackage{graphicx}   % Essential for adding images to you document.

\usepackage{sectsty}
\sectionfont{\large}
\subsectionfont{\normalsize}
\subsubsectionfont{\normalsize \it}

\usepackage{caption}
\captionsetup{labelsep=period}

\pagestyle{fancy} % allows you to use the header and footer commands 

\raggedright
\begin{document}

\setlength{\parindent}{0in} % Amount of indentation at the first line of a paragraph.

\pagestyle{fancy} \lhead{Revealing the Physics of Galactic Winds with Petascale GPU Simulations} \rhead{Brant Robertson} \renewcommand{%
\headrulewidth}{0.0pt}



\centering {\bf Curriculum Vitae ({\emph{2-page limit}})}\\
{\bf Evan Schneider}\\
{\bf Dept. of Astronomy, Princeton University, eschneider@as.arizona.edu} \smallskip

\begin{flushleft} {\bf Professional Preparation}
{\parindent 16pt

PhD, Astronomy \& Astrophysics, University of Arizona, 2017 \\ 
MS, Astronomy, University of Arizona, 2012 \\ 
BA, Physics \& Mathematics, Bryn Mawr College, 2010 \\ 
}

\vspace{.04in}
{\bf Appointments}
{\parindent 16pt

2016--2017 Junior Research Specialist, University of California, Santa Cruz \\ 
2014--2016 Graduate Research Assistant, University of Arizona \\ 
2011--2014 NSF Graduate Research Fellow, University of Arizona \\ 
2010--2011 Steward Observatory Graduate Fellow, University of Arizona \\ 
}

\vspace{.04in}
{\bf Five Publications Most Relevant to This Proposal}
\vspace{-6pt}
\begin{enumerate} \itemsep1pt \parskip0pt \parsep0pt
\item "Hydrodynamical Coupling of Mass and Momentum in Multiphase Galactic Winds", Schneider, E. \& Robertson, B. \textit{The Astrophysical Journal}, {\bf 834}, 144 (2017)\\ 
\item "CHOLLA: A New Massively Parallel Hydrodynamics Code for Astrophysical Simulation", Schneider, E. \& Robertson, B. \textit{Astrophysical Journal Supplements}, {\bf 217}, 24 (2015)\\ 
\end{enumerate} 

\vspace{-6pt}
{\bf Research Interests and Expertise}
{\parindent 16pt
CoI Schneider's research interests center on the ways in which hydrodynamic processes affect galaxy formation and evolution, particularly the effects of stellar feedback. As the primary developer of the GPU-based astrophysics code \textit{Cholla}, Schneider is an expert in hydrodynamical simulation methodology. Given the dynamic range required in cosmological simulations, many baryonic processes remain unresolved. Schneider's Ph.D. thesis and ongoing work consist of using the code \textit{Cholla} to produce petascale astrophysical simulations that reveal previously unknown details of galactic structure, including the turbulent interstellar medium and galactic outflows. Schneider served as CoI of the OLCF DD Project AST107 ``Scaling the GPU-enabled Hydrodynamics Code Cholla to the Power of Titan" and DD Project AST119 ``Extending the Physics of the GPU-Enabled CHOLLA Code to the Power of Titan", and is Co-PI of the OLCF INCITE Project AST125 ``Revealing the Physics of Galactic Winds with Petascale GPU Simulaions".
}

\vspace{.04in}
{\bf Synergistic Activities}
\vspace{-6pt}
\begin{enumerate} \itemsep1pt \parskip0pt \parsep0pt
\item Primary developer and maintainer of the astrophysical hydrodynamics code, \textit{Cholla}. \\ 
\item Presented at various University of Arizona events emphasizing the utility of HPC systems. \\ 
\item Attended the 2017 OLCF User Meeting in Oak Ridge, TN. \\ 
\item Advocate for improving the representation of minorities in the HPC community. \\
\item Referee for ApJ and MNRAS. \\
\end{enumerate} 

\vspace{-6pt}
{\bf Collaborators ({\emph{past 5 years including name and current institution}})} \\
{\parindent 16pt
Robertson, B. E., University of California, Santa Cruz \\
Impey, C. D., University of Arizona \\
Trump, J. R., Pennsylvania State University \\
Salvato, M., Max Planck Institute for Extraterrestrial Physics \\
Koekemoer, A., Space Telescope Science Institute \\
Ellis, R. S., European Southern Observatory \\
McLure, R. J., University of Edinburgh \\
Dunlop, J. S., University of Edinburgh \\
Ono, Y., University of Tokyo \\
Schenker, M., PDT Partners \\
Ouchi, M., University of Tokyo \\
Bowler, R., University of Edinburgh \\
Rogers, A., University of Edinburgh \\
Curtis-Lake, E., University of Edinburgh \\
Charlot, S., Institut d? Astrophysique de Paris \\
Stark, D. P., University of Arizona \\
Furlanetto, S., University of California, Los Angeles Cirasuolo, M., University of Edinburgh \\
Wild, V., University of St. Andrews \\
Targett, T. A., Sonoma State University \\
Shimasaku, K., University of Tokyo \\
Dayal, P., University of Groningen \\
Dupree, A. K., Harvard-Smithsonian Center for Astrophysics \\
Brickhouse, N. S., Harvard-Smithsonian Center for Astrophysics \\
Cranmer, S. R., Harvard-Smithsonian Center for Astrophysics \\
Luna, G. J., Instituto de Astronomia y Fisica del Espacio \\
Bessell, M. S., Australian National Observatory, Canberra \\
Bonanos, A., National Observatory of Athens \\
Crause, L. A., South African Astronomical Observatory \\
Lawson, W. A., University of New South Wales \\
Mallik, S. V., Indian Institute of Astrophysics \\
Schuler, S. C., National Optical Astronomy Observatory \\
Ransom, S. M., National Radio Astronomy Observatory \\
Beckmann, P. A., Bryn Mawr College \\
}


\end{flushleft}

\end{document}
