\documentclass[11pt,letterpaper,english]{article}
\usepackage[T1]{fontenc} % Standard package for selecting font encodings
\usepackage{txfonts} % makes spacing between characters space correctly
\usepackage{xcolor} % Driver-independent color extensions for LaTeX and pdfLaTeX.
%\usepackage{blindtext} % To create text
%\usepackage{mdwlist} % mdwlist for compact enumeration/list items 
%\usepackage[pagestyles]{titlesec} % related with sections—namely titles, headers and contents
\usepackage{fancyhdr} % header footer placement

\usepackage[top=1in, bottom=1in, left=1in, right=1in] {geometry} % Margins
\usepackage{graphicx}   % Essential for adding images to you document.

\usepackage{sectsty}
\sectionfont{\large}
\subsectionfont{\normalsize}
\subsubsectionfont{\normalsize \it}

\usepackage{caption}
\captionsetup{labelsep=period}

\pagestyle{fancy} % allows you to use the header and footer commands 

\raggedright
\begin{document}

\setlength{\parindent}{0in} % Amount of indentation at the first line of a paragraph.

\pagestyle{fancy} \lhead{Revealing the Physics of Galactic Winds with Petascale GPU Simulations} \rhead{Brant Robertson} \renewcommand{%
\headrulewidth}{0.0pt}



\centering {\bf Curriculum Vitae ({\emph{2-page limit}})}\\
{\bf Brant Robertson}\\
{\bf Dept. of Astronomy and Astrophysics, University of California, Santa Cruz, brant@ucsc.edu} \smallskip

\begin{flushleft} {\bf Professional Preparation}
{\parindent 16pt

PhD, Astronomy, Harvard University, 2006 \\ 
MA, Astronomy, Harvard University, 2003 \\ 
BS, Physics \& Astronomy, University of Washington, 2001 \\ 
}

\vspace{.04in}
{\bf Appointments}
{\parindent 16pt

2011--2015 Assistant Professor, University of Arizona \\ 
2009--2011 Hubble Fellow, California Institute of Technology \\ 
2006--2009 Spitzer and Institute Fellow, Kavli Institute for Cosmological Physics\\ 
}
{\parindent 70pt
and Enrico Fermi Institute, University of Chicago \\ 
}

\vspace{.04in}
{\bf Five Publications Most Relevant to This Proposal}
\vspace{-6pt}
\begin{enumerate} \itemsep1pt \parskip0pt \parsep0pt
\item "Hydrodynamical Coupling of Mass and Momentum in Multiphase Galactic Winds", Schneider, E. \& Robertson, B. \textit{The Astrophysical Journal}, {\bf 834}, 144 (2017)\\ 
\item "CHOLLA: A New Massively Parallel Hydrodynamics Code for Astrophysical Simulation", Schneider, E. \& Robertson, B. \textit{Astrophysical Journal Supplements}, {\bf 217}, 24 (2015)\\ 
\item "Adiabatic Heating of Contracting Turbulent Fluids", Robertson, B. \& Goldreich, P. \textit{Astrophysical Journal Letters}, {\bf 750}, 31 (2012)\\ 
\item "Computational Eulerian hydrodynamics and Galilean Invariance", Robertson, B., et al. \textit{Monthly Notices of the Royal Astronomical Society}, {\bf 401}, 2463 (2010)\\ 
\item "Molecular Hydrogen and Global Star Formation Relations in Galaxies", Robertson, B. \& Kravtsov, A. \textit{Astrophysical Journal}, {\bf 680}, 1083 (2008)\\ 
\end{enumerate} 

\vspace{-6pt}
{\bf Research Interests and Expertise}
{\parindent 16pt ~\\
Brant Robertson's research interests include theoretical topics related to galaxy formation, dark matter, hydrodynamics, and numerical simulation methodologies. His numerical simulation expertise includes published first author papers on cosmological n-body simulations, smoothed particle hydrodynamics simulations of galaxy formation, and simulations of supersonic turbulence in astrophysical settings. He served as Evan Schneider's PhD advisor during the development of {\it Cholla}. Robertson served as PI of the OLCF DD Project AST107 \textit{``Scaling the GPU-enabled Hydrodynamics Code Cholla to the Power of Titan"}, DD Project AST119 \textit{``Extending the Physics of the GPU-Enabled CHOLLA Code to the Power of Titan"}, and INCITE Project AST125 \textit{``Revealing the Physics of Galactic Winds with Petascale GPU Simulations''}.
}

\vspace{.04in}
{\bf Synergistic Activities}
\vspace{-6pt}
\begin{enumerate} \itemsep1pt \parskip0pt \parsep0pt
\item Simulated the future galaxy merger of Milky Way and Andromeda for use in educational videos: shown
in planetaria shows, the NOVA ``Monster of the Milky Way'' special, and the COSMOS TV show on FOX;
viewed by tens of millions of people.\\
\item Co-Chair of the Large Synoptic Survey Telescope Galaxies Science Collaboration, a group of more
than 100 scientists interested in using the LSST data to learn about galaxy formation and evolution.\\
\item Steering Committee of the joint NIRCam-NIRSpec James Webb Space Telescope GTO program that
will perform the first ultra deep extragalactic survey with JWST in the first year of operations.\\
\item Member of the Hubble Deep Fields Initiative Committee that recommended the Frontier Fields
community program with Hubble Space Telescope surveying strong gravitational lens clusters.\\
\item Referee for Nature, ApJ, ApJL, ApJS, MNRAS, MNRAS Letters, Astronomy and Astrophysics, and the
Journal of Fluid Mechanics-Rapid.\\
\end{enumerate} 

\vspace{-6pt}
{\bf Collaborators ({\emph{past 5 years including name and current institution}})} 
{\parindent 16pt
}


\end{flushleft}

\end{document}
